\documentclass[11pt]{article}

\usepackage{bm} %Makes symbols bold'
\usepackage{amsmath}
\usepackage{amsthm}
\usepackage{dsfont}
\usepackage{amssymb} %Creates new symbols'
\usepackage{cancel} 
\usepackage{enumerate}
\usepackage{braket}


\title{Molecular Dynamics}
\author{Georgios Smyridis}
\date{2023}

\begin{document}
\maketitle


\section{Methods}

\subsection{Molecular Dynamics Simulations}

In my script I created a class named $MolecularDynamics$ which includes all the attributes of the system as well as the methods needed for the simulation. Every Molecular Dynamics simulation consists of the following fundamental steps.
\begin{enumerate}
	\item \textbf{Start with an initial configuration} $\Gamma_0$, which includes determining the initial positions and velocities of all the particles.

	 In our case, at the start of the simulation, 256 particles (balls of radius 1) are located on the lattice points of an FCC lattice with periodic boundary conditions. We can then specify the density $\rho$ of the system with the use of the method \textit{set\_density($\rho$)}, which rescales the dimensions of the box as well as the intermolecular distances accordingly. 
	 
	 The initial velocities are chosen randomly and uniformly from the range $[-1, 1]$ and then are shifted 
	 	so that the total momentum of the system is zero and then rescaled, so that the total kinetic energy corresponds to the set initial temperature. This can be seen by noticing that
	 	\begin{align}
	 		E_{kin}=&\sum_i\frac{m v_i^2}2=\frac{3k_BT}2\nonumber\\
	 		&\frac1N\sum_i v_i^2=3k_BT \label{kinetic_energy_temperature}
	 	\end{align}
	where we have taken $m=1$.
	
	\item \textbf{Calculate the forces acting on every particle}. Assuming that every particle interacts with the closest picture of every other particle in the box, we loop over every particle and sum all the forces component-wise. In explicit, the force that the particle $i$ feels is
		\begin{align}
			\bm{F}_i=\sum_{j\neq i}\bm{F}(r_{ij})
		\end{align}
		where $r_{ij}$ is the distance between particles $i$ and $j$ taking into account the periodic conditions of the box. Now we have to determine the specific form of $F(r_{ij})$ for our system. We are studying a Lennard-Jones fluid and the potential is given by
		\begin{align}
			u(r_{ij})=
			\left\{\begin{aligned}
				&4\epsilon\bigg[\bigg(\frac{\sigma}{r_{ij}}\bigg)^{12}-\bigg(\frac{\sigma}{r_{ij}}\bigg)^6\ \bigg]-\epsilon_{cut}\qquad r_{ij}\geq r_{cut}\\
				&0 \qquad\qquad\qquad\qquad\qquad\qquad\qquad\quad  r_{ij}<r_{cut}
			\end{aligned}
			\right.
		\end{align}
		where
		\begin{align}
			e_{cut} = 4\epsilon \bigg[\bigg(\frac{\sigma}{r_{cut}}\bigg)^{12}-\bigg(\frac{\sigma}{r_{cut}}\bigg)^{6}\bigg]
		\end{align}
		Then, the force that acts on particle $i$ because of particle $j$ is
		\begin{align}
			f(r)&=-\nabla u(r)=-\sum_{i=1}^3\frac{\partial u(r)}{\partial x_i}\hat x_i=-\sum_{i=1}^3\frac{\partial r}{\partial x_i}\frac{\partial u(r)}{\partial r}\hat x_i\nonumber\\
			&=-\frac{\partial u(r)}{\partial r}\hat r=\frac{48\epsilon}{\sigma}\bigg[\bigg(\frac{\sigma}{r_{ij}}\bigg)^{13}-\bigg(\frac{\sigma}{r_{ij}}\bigg)^7\ \bigg]
		\end{align}
		In the simulation we have taken $\epsilon = \sigma = 1$.
		
		\item \textbf{Integration of Newton's equations to obtain new positions and velocities.} We are going to use the so called 'velocity Verlet' algorithm which updates the positions according to the formula
			\begin{align}
				r_i(t+\Delta t) = r_i(t) + v_i(t)\Delta t + \frac{f_i(t)}{2m}\Delta t^2 + \mathcal{O}(\Delta t^4)
				\label{position_update}
			\end{align}
			After we have calculated the new positions we have to enforce the periodic boundary conditions, that is if one component of the positions is outside the box, we displace is by the length of the box to the direction so that it is in 
			
			and the velocities according to:
		\begin{align}
			v_i(t+\Delta t)=v_i(t)+\frac{f_i(t+\Delta t)+f_i(t)}{2m}\Delta t + \mathcal{O}(\Delta t^2)
			\label{velocity_update}
		\end{align}
		In this integration scheme we calculate the positions and the velocities at equal times. The calculation of the new velocities requires first the calculation of the new positions and the forces at these new positions.
		The velocities then can be used to calculate the kinetic energy, and therefore, the instantaneous temperature with \eqref{kinetic_energy_temperature}. It is important that \eqref{position_update} and \eqref{velocity_update} imply that the total linar momentum of the system is zero at all times and that the total energy is conserved, with the total energy being the sum of the total kinetic energy and the total potential energy
		\begin{align}
			E_{pot}=\sum_{i<j}u(r_{ij})
		\end{align}
\end{enumerate}


\subsection{NVE Ensemble}


\subsection{NVT Ensemble}


In the canonical ensemble the temperature of the system is kept constant by being in contact with a heat bath. In molecular dynamics simulations there are a few ways to enforce this condition, but the one I have used is the 'Andersen thermostat'. Under this scheme, the coupling to a heat bath is represented by stochastic forces that act occasionally on randomly selected particles of the system. These forces transfer the system from one constant-energy shell to another, while between them the system evolves at constant energy according to the Newton's equations of motion (so, like in the NVE ensemble). These stochastic forces ensure that the system visits all constant-energy shells according to their Boltzmann weight. The steps to perform such a simulation are
\begin{enumerate}
	\item 
\end{enumerate}


\subsection{Diffusion}

Diffusion is, in layman's terms, the phenomenon of 'washing out' inhomogeneities. Fick's law states that the flux $\bm{j}(\bm{r},t)$ of a set of labeled particles at position $\bm{r}$ and time $t$, is proportional to (negative) the gradient of concentration with the factor of proportionality being the self-diffusion coefficient. When the number of particles is fixed, i.e. they are not created or destroyed, we can combine it with the equation of continuity to arrive a differential equation for the concentration. Then, we can make use of this relation to find an expression for the average-squared distance traveled by the labeled particles in time $t$, and ultimately, a formula that calculates the self-diffusion coefficient, namely,
\begin{align}
	D=\frac1d\int_0^\infty dt\braket{\bm{v}(0)\cdot\bm{v}(t)},
\end{align}
where $d$ is the dimensions of the system (for us, $d=3$). Hence, we have an expression linking the self-diffusion coefficient to the velocity autocorrelation function. This is an example of a Green-Kubo formula.





\end{document}
